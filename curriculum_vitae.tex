% LaTeX Curriculum Vitae Template
%
% Copyright (C) 2004-2009 Jason Blevins <jrblevin@sdf.lonestar.org>
% http://jblevins.org/projects/cv-template/
%
% You may use use this document as a template to create your own CV
% and you may redistribute the source code freely. No attribution is
% required in any resulting documents. I do ask that you please leave
% this notice and the above URL in the source code if you choose to
% redistribute this file.

\documentclass[letterpaper]{article}

\usepackage{hyperref}
\usepackage{geometry}

% Comment the following lines to use the default Computer Modern font
% instead of the Palatino font provided by the mathpazo package.
% Remove the 'osf' bit if you don't like the old style figures.
\usepackage[utf8]{inputenc}
\usepackage[T1]{fontenc}
\usepackage[sc,osf]{mathpazo}
  
\usepackage[explicit]{titlesec}
\titleformat{\section}{\Large}{\thesection}{1zw}{\titleline[l]{}\titlerule}

% Set your name here
\def\name{Yuta Tokuoka}

% Replace this with a link to your CV if you like, or set it empty
% (as in \def\footerlink{}) to remove the link in the footer:
\def\footerlink{https://fun.bio.keio.ac.jp/\~tokuoka/}

% The following metadata will show up in the PDF properties
\hypersetup{
  colorlinks = true,
  urlcolor = black,
  pdfauthor = {\name},
  pdfkeywords = {bioinformatics, statistics, mathematics},
  pdftitle = {\name: Curriculum Vitae},
  pdfsubject = {Curriculum Vitae},
  pdfpagemode = UseNone
}

\geometry{
  body={6.5in, 8.5in},
  left=1.0in,
  top=1.25in
}

% Customize page headers
\pagestyle{myheadings}
\markright{\name}
\thispagestyle{empty}

% Custom section fonts
\usepackage{sectsty}
\sectionfont{\rmfamily\mdseries\Large}
\subsectionfont{\rmfamily\mdseries\itshape\large}
\usepackage{url}

% Other possible font commands include:
% \ttfamily for teletype,
% \sffamily for sans serif,
% \bfseries for bold,
% \scshape for small caps,
% \normalsize, \large, \Large, \LARGE sizes.

% Don't indent paragraphs.
\setlength\parindent{0em}

% Make lists without bullets
\renewenvironment{itemize}{
  \begin{list}{}{
    \setlength{\leftmargin}{1.5em}
  }
}{
  \end{list}
}

\begin{document}

% Place name at left
{\huge \name}

% Alternatively, print name centered and bold:
%\centerline{\huge \bf \name}

\vspace{0.25in}

\begin{minipage}{0.45\linewidth}
  \href{http://www.unc.edu/}{Preferred Networks, Inc.} \\
  Otemachi Bldg., 1-6-1 Otemachi, \\
  Chiyoda-ku, Tokyo \\
  100-0004 Japan
\end{minipage}
\begin{minipage}{0.45\linewidth}
  \begin{tabular}{ll}
   Phone: & +81 90 2222 7014 \\
   Email: & \href{mailto:bowbrand1227@gmail.com}{\tt bowbrand1227@gmail.com} \\
   Homepage: & \href{https://fun.bio.keio.ac.jp/\~tokuoka/}{\tt \url{https://fun.bio.keio.ac.jp/\~tokuoka/}} \\
   GitHub: & \href{https://github.com/tokkuman}{\tt \url{https://github.com/tokkuman}} \\
  \end{tabular}
\end{minipage}

\vspace{0.6cm}

% \section*{Interest}

% He received the B.E. and M.E. degrees in Biosciences and Informatics
% from Keio University, in 2016 and 2018, respectively. He joined the
% School of Fundamental Science and Technology at Keio University in 2018,
% where he is currently a Ph.D. candidate under the supervision of
% Dr. Akira Funahashi. His research interests include computer vision,
% machine learning, and developmental biology.


\section*{\bf Affiliation}
\vspace{-0.6cm}
\hrulefill
% \vspace{-0.3cm}

\begin{itemize}
 \item {\bf Researcher, Preferred Networks, Inc.}, Research and Development division, Healthcare and Wellness / Life and Materials Science, 2022--present.
 \item {\bf Researcher, Keio University}, Biosciences and Informatics, Keio University, 2022--present.
       \\
\end{itemize}


\section*{\bf Education}
\vspace{-0.6cm}
\hrulefill
% \vspace{-0.3cm}

\begin{itemize}
 \item {\bf Ph.D. in Engineering}, Graduate School of Science and Technology, Keio University, 2018--2022.
 \item {\bf M.E}, Biosciences and Informatics, Keio University, 2016--2018.
 \item {\bf B.E}, Biosciences and Informatics, Keio University, 2012--2016.
       \\
\end{itemize}


\section*{\bf Professional Experience}
\vspace{-0.6cm}
\hrulefill

\begin{itemize}
 \item Mentor for Google Summer of Code 2020 \& 2021
 \item Preferred Networks, Inc. Summer Intern and Part-time Engineer 2018--2019
       \\
       %An Inductive Transfer Learning Approach using Cycle-consistent
       %Adversarial Domain Adaptation with Application to Brain Tumor
       %Segmentation
\end{itemize}


\section*{\bf Publications and Preprints}
\vspace{-0.6cm}
\hrulefill
\vspace{-0.3cm}

\subsection*{\bf Journal Articles}

\begin{itemize}
 \item Jin Komuro, \underline{Yuta Tokuoka}, Tomohisa Seki, Dai
       Kusumoto, Hisayuki Hashimoto, Toshiomi Katsuki, Takahiro
       Nakamura, Yohei Akiba, Thukaa Kuoka, Mai Kimura, Takahiro Yamada,
       Keiichi Fukuda, Akira Funahashi, Shinsuke Yuasa, ``Development of
       non-bias phenotypic drug screening for cardiomyocyte hypertrophy
       by image segmentation using deep learning,'' {\itshape
       Biochemical and Biophysical Research Communications}, 632,
       pp.181--188, (2022)

 \item \underline{Yuta Tokuoka}, Takahiro G Yamada, Daisuke Mashiko,
       Zenki Ikeda, Tetsuya J Kobayashi, Kazuo Yamagata, Akira
       Funahashi, ``An explainable deep learning-based algorithm with an
       attention mechanism for predicting the live birth potential of
       mouse embryos,'' {\itshape Artificial Intelligence in Medicine},
       134, 102432, (2022)

 \item Hazumi Kubota, \underline{Yuta Tokuoka}, Takahiro G Yamada, Akira
       Funahashi, ``Symbolic Integration by Integrating Learning Models
       With Different Strengths and Weaknesses,'' {\itshape IEEE
       Access}, 10, pp.470000--47010, (2022)

 \item \underline{Yuta Tokuoka}, Takahiro G Yamada, Daisuke Mashiko,
       Zenki Ikeda, Noriko F Hiroi, Tetsuya J Kobayashi, Kazuo Yamagata,
       Akira Funahashi, ``3D convolutional neural networks-based
       segmentation to acquire quantitative criteria of the nucleus
       during mouse embryogenesis,'' {\itshape npj Systems Biology and
       Applications}, 6.32, (2020)

 \item Chikahiro Imashiro, \underline{Yuta Tokuoka}, Kaito Kikuhara,
       Takahiro G Yamada, Kenjiro Takemura, Akira Funahashi, ``Direct
       Cell Counting Using Macro-Scale Smartphone Images of Cell
       Aggregate,'' {\itshape IEEE Access}, 8, pp.170033-170043, (2020)

 \item Maya Ooka, \underline{Yuta Tokuoka}, Shori Nishimoto, Noriko
       F. Hiroi, Takahiro G. Yamada, Akira Funahashi, ``Deep Learning
       for Non-Invasive Determination of the Differentiation Status of
       Human Neuronal Cells by Using Phase-Contrast Photomicrographs,''
       {\itshape Applied Sciences} 9.24 (2019): 5503.

 \item Shori Nishimoto, \underline{Yuta Tokuoka}, Takahiro G. Yamada,
       Noriko F. Hiroi, Akira Funahashi, ``Predicting the future
       direction of cell movement with convolutional neural networks,''
       {\itshape PloS one} 14.9 (2019).
\end{itemize}


\subsection*{\bf Proceedings}

\begin{itemize}
 \item \underline{Yuta Tokuoka}, Shuji Suzuki, Yohei Sugawara,
       ``An Inductive Transfer Learning Approach using Cycle-consistent Adversarial Domain Adaptation with Application to Brain Tumor Segmentation,''
       {\itshape 2019 6th International Conference on Biomedical and Bioinformatics Engineering}, Accepted.
\end{itemize}


% \subsection*{\bf Preprints}

% \begin{itemize}
%  \item \underline{Yuta Tokuoka}, Takahiro G. Yamada, Noriko F. Hiroi,
%        Tetsuya J. Kobayashi, Kazuo Yamagata, Akira Funahashi,
%        ``Convolutional Neural Network-Based Instance Segmentation Algorithm
%        to Acquire Quantitative Criteria of Early Mouse Development,''
%        {\itshape BioRxiv} (2018): 324186.
% \end{itemize}


\subsection*{\bf Books/Reviews}

\begin{itemize}
 \item 舟橋啓, \underline{徳岡雄大}, 今城哉裕, (2021), 遠心が1回で済む! スマートフォンによる細胞計数, 実験医学, Vol.39 No.8, pp. 1283-1290.
 \item 舟橋啓, 板橋正寛, 大岡麻耶, 小林佑也, 斎藤昴哉, \underline{徳岡雄大}, 中村隆之, 中村善次, 西本勝利, 広井賀子, 本室美貴子, 山崎孔敬, 山田貴大, 山中龍, (2020), 数でとらえる細胞生物学, 羊土社, 4章, pp. 188-252.
 \item 舟橋啓, \underline{徳岡雄大}, 山田貴大, (2020), Google Colaboratory入門―機械学習を体験しよう, 実験医学増刊, Vol.38 No.20, pp. 16-35.
 \item 山田貴大, \underline{徳岡雄大}, 尾関光徳, 井伊海人, 広井賀子, 舟橋啓, (2020), クラシフィケーションの原理と生物・医療への応用, 実験医学増刊, Vol.38 No.20, pp. 109-117.
 \item \underline{徳岡雄大}, 山田貴大, 舟橋啓, (2020), 機械学習によるバイオイメージセグメンテーション, 実験医学増刊, Vol.38 No.20, pp. 134-141.
 \item 舟橋啓, \underline{徳岡雄大}, 大岡麻耶, 西本勝利, 山田貴大, 広井賀子, (2019), 画像解析と深層学習を用いた診断の基礎, 病理と臨床 , Vol. 37, No.7. pp. 631-635
\end{itemize}


% \subsection*{\bf Reviews}

% \begin{itemize}
%  \item 舟橋啓, \underline{徳岡雄大}, 大岡麻耶, 西本勝利, 山田貴大, 広井賀子,
%        「画像解析と深層学習を用いた診断の基礎」, 病理と臨床 37(7): 631-635, 2019.
% \end{itemize}



\section*{\bf Report of Teaching and Training}
\vspace{-0.6cm}
\hrulefill

\begin{itemize}
 \item \underline{徳岡雄大}.「不妊治療に資する深層学習を用いた初期胚定量
       評価手法の開発」群馬大学数理データ科学教育研究センター主催第1回レ
       ギュラトリーサイエンスセミナー群馬, 2019年9月5日
 \item 舟橋啓, \underline{徳岡雄大}.「新学術領域研究・学術研究支援基盤形
       成 「先端バイオイメージング支援プラットフォーム」AIによる生物画像
       解析トレーニングコース」機械学習による画像分類, 熊本, 2019年8月29
       日 \\ (https://github.com/funalab/ABiSTC) \\
\end{itemize}


\section*{\bf Invited Presentations}
\vspace{-0.6cm}
\hrulefill

\begin{itemize}
 \item \underline{徳岡雄大}, 山田貴大, 増子大輔, 池田善喜, 広井賀子, 小
       林徹也, 山縣一夫, 舟橋啓.「深層学習が捉えた胚の動的形態変化を利用
       したマウス胚出生予測」, 生殖若手の会 第8回大会, 東京, 2022年9月11
       日
 \item \underline{徳岡雄大}, 山田貴大, 増子大輔, 池田善喜, 広井賀子, 小
       林徹也, 山縣一夫, 舟橋啓.「深層学習が捉えた胚の動的形態変化を利用
       したマウス胚出生予測」,日本顕微鏡学会第78回学術講演会, 福島, 2022
       年5月13日
 \item \underline{徳岡雄大}, 山田貴大, 広井賀子, 小林徹也, 山縣一夫, 舟
       橋啓.「不妊治療に資する深層学習を用いた初期胚定量評価手法の開発」,
       第18回日本再生医療学会総会, 兵庫, 2019年3月23日
 \item \underline{Yuta Tokuoka}, Noriko F. Hiroi, Tetsuya J. Kobayashi,
       Kazuo Yamagata, Akira Funahashi. ``Segmenting four-dimensional
       fluorescence microscopic image using Convolutional Neural
       Network,'' 18th International Conference on Systems Biology,
       Blacksburg, Virginia, USA, Aug, 2017 \\
\end{itemize}


\section*{\bf Software Developments}
\vspace{-0.6cm}
\hrulefill

\begin{itemize}
 \item {\bf Classification of Neuronal Differentiation (CoND)} \\
       https://github.com/funalab/CoND
 \item {\bf PredictMovingDirection} \\
       https://github.com/funalab/PredictMovingDirection
 \item {\bf Quantitative Criteria Acquisition Network (QCANet)} \\
       https://github.com/funalab/QCANet
 \item {\bf Grouping Neural Network} \\
       https://github.com/tokkuman/GroupingNN
       \\
\end{itemize}


\section*{\bf Honors and Fellowships}
\vspace{-0.6cm}
\hrulefill

\begin{itemize}
 \item JSPS Research Fellowship for Young Scientists (DC2), 2019
 \item Amano Institute of Technology fellowship, 2018
 \item GTC Japan 2017 Best poster award finalist, 2017
       \\
\end{itemize}


\section*{\bf Technical and Personal Skills}
\vspace{-0.6cm}
\hrulefill
\vspace{-0.3cm}

\subsection*{\bf Coursework}
Experimental/theoretical cell biology, developmental biology, machine
learning, computer vision, statistics, nonlinear dynamics,
ordinary/partial differential equations

\subsection*{\bf Computing}
Proficient: Python, C/C++, Git/GitHub/GitLab, Chainer, PyTorch \\
Basic: Docker, Java, JavaScript, Tensorflow, Keras

\bigskip

% Footer
\begin{center}
  \begin{footnotesize}
    Last updated: \today \\
    \href{\footerlink}{\texttt{\url{\footerlink}}}
  \end{footnotesize}
\end{center}

\end{document}
